\documentclass[hyperref={colorlinks = true, urlcolor = blue},pdf]{beamer}
\mode<presentation>{}

%% packages
%%%% unicode
\usepackage[utf8]{inputenc}
%%%% math
\usepackage{amsmath}
\usepackage{amssymb}
\usepackage{stmaryrd}

%% macros
\newcommand{\Semantic}[1]{\llbracket {#1} \rrbracket}
\newcommand{\SuchThat}{\ \vert\ }
\newcommand{\Size}[1]{\vert {#1} \vert}

%% preamble
\title{Recursive Types}
\subtitle{Programming Languages Reading Group, Indiana University}
\author{Kartik Sabharwal}
\date{2022-02-01}

\begin{document}
%% title frame
\begin{frame}
  \titlepage
\end{frame}
%% normal frame
\begin{frame}{Types as Sets of Values}
  \begin{align*}
    \Semantic{\text{Integer}} =\ &
    \{ \ldots, -2, -1, 0, 1, 2, \ldots \} \\
    %
    \Semantic{A \cup B} =\ &
    \Semantic{A} \cup \Semantic{B} \\
    %
    \Semantic{A \cap B} =\ &
    \Semantic{A} \cap \Semantic{B} \\
    %
    \Semantic{A \times B} =\ &
    \Semantic{A} \times \Semantic{B} \\
    %
    \Semantic{A + B} =\ &
    \{ \text{inj}_1(a) \ \vert\ a \in \Semantic{A} \} \cup
    \{ \text{inj}_2(b) \ \vert\ b \in \Semantic{B} \} \\
    %
    \Semantic{A \rightarrow B} =\ &
    \Semantic{B}^{\Semantic{A}}\
    (\text{the set of functions from}\ A\ \text{to}\ B)
  \end{align*}
  %
  \textit{Trivia.} If $X$ and $Y$ are finite sets,
  %
  \begin{align*}
    \Size{X \times Y} =\ & \Size{X} \times \Size{Y} \\
    \Size{X + Y} =\ & \Size{X} + \Size{Y} \\
    \Size{Y^X} =\ & {\Size{Y}}^{\Size{X}}
  \end{align*}
\end{frame}
%%
\begin{frame}[fragile]{Types as Sets of Values -- Examples}
\begin{verbatim}
#lang typed/racket/base

(struct (S) inj1 ([s : S]))
(struct (T) inj2 ([t : T]))
(define-type (Sum S T) (U (inj1 S) (inj2 T)))

(define-type A (U False 1 2)) 
(define-type B Boolean)

(define ex1 : (U A B) 2)
(define ex2 : (Intersection A B) #f)
(define ex3 : (Pair A B) (cons 1 #t))
(define ex4 : (Sum A B) (inj1 (ann 2 A)))
(define ex5 : (-> A B) (lambda ([a : A]) (if a #t a)))
\end{verbatim}
\end{frame}
%%
\begin{frame}[fragile]{Recursive Types -- Motivation 1}
  A lot of objects that we want to model using programming languages have a
  self-referential structure. Natural numbers, lists, binary trees and abstract
  syntax trees are popular examples.

\begin{verbatim}
(define-type Natural^ (Sum Null Natural^))

(define ex6 : Natural^ (inj2 (inj2 (inj1 null))))
;; * * * * * * * * * * * * * * * * * * * * * * * * *
(define-type IntList (U Null (Pair Integer IntList)))

(define ex7 : IntList null)
(define ex8 : IntList (cons 1 (cons 5 (cons 4 null))))
\end{verbatim}
\end{frame}
%%
\begin{frame}[fragile]{Recursive Types -- Motivation 2}
\begin{verbatim}
(struct Leaf ())
(struct (T) Node ([left : T]
                  [value : Integer]
                  [right : T]))
(define-type Tree (U Leaf (Node Tree)))

(define ex9 : Tree (Node (Leaf) 1 (Leaf)))
(define ex10 : Tree (Node ex9 0 ex9))
;; * * * * * * * * * * * * * * * * * * * * * * * * *
(define-type IntFun (U Integer (-> Integer IntFun)))

(define ex11 : IntFun
  (lambda ([x : Integer])
    (if (> x 0) x ex11)))
\end{verbatim}
\end{frame}
%%
\begin{frame}[fragile]{Recursive Types -- Motivation 3}
\begin{verbatim}
(struct Lit ([b : Boolean]))
(struct Var ([s : Symbol]))
(struct (BE) And ([beL : BE] [beR : BE]))
(struct (BE) Or ([beL : BE] [beR : BE]))
(struct (BE) Not ([be : BE]))
(define-type BoolExp
  (U Lit Var (And BoolExp) (Or BoolExp) (Not BoolExp)))

(define ex12 : BoolExp (Not (And (Lit #t) (Var 'x))))
\end{verbatim}
We ought to be able to visualize these types as sets when we want to prove
properties about them.
\end{frame}
%%
\begin{frame}{Recursive Type $\rightarrow$ Generating Function}
  We defined a recursive type intended to represent the natural numbers via an
  equation.
  %
  \begin{equation*}
    \text{Natural}\ =\ \text{Null} + \text{Natural}
  \end{equation*}
  %
  We can confine the ``unknown'' in the equation to the left-hand side using a
  $\mu$ variable binder.
  %
  \begin{equation*}
    \text{Natural}\ =\ \mu X .\ \text{Null} + X
  \end{equation*}
  %
  TAPL defines $\Semantic{\text{Natural}}$ as the ``greatest fixed-point'' of
  the following function on sets of values. We can call it the
  ``generating function'' for Natural.
  %
  \begin{equation*}
    F(X) = \{ \text{inj1(null)} \} \cup \{ \text{inj2}(x) \SuchThat x \in X \}
  \end{equation*}
\end{frame}
%%
\begin{frame}{Fixed-Point}
  A ``fixed-point'' of a function is a value that doesn't change when we apply
  the function to it.
  \medskip

  Any set $X$ such that $F(X) = X$ is a fixed-point of $F$
  \medskip

  If $F$ has two fixed-points $X_1$ and $X_2$,
  $X_1 \subseteq X_2 \implies X_1 \leq X_2$.
\end{frame}
%%
\begin{frame}{Least Fixed Point}
  Kleene's Fixed-Point Theorem shows us how to compute the least fixed-point of
  $F$ -- repeatedly apply $F$ to $\varnothing$ and perform a union over all
  elements in this sequence.
  %
  \begin{align*}
    F^1(\varnothing) =\ & \{ \text{inj1(null)} \} \\
    F^2(\varnothing) =\ & \{ \text{inj1(null), inj2(inj1(null))} \} \\
    \cdots & \\
    F^n(\varnothing) =\ &
    \{ \text{inj1(null)}, \ldots, {\text{inj2}}^{(n-1)}\text{(inj1(null))} \}
  \end{align*}
  %
  Applying $F$ to the set $\cup_{i=0}^\infty F^i(\varnothing)$ doesn't add or
  remove any elements from it. We have our least fixed-point.
  \medskip

  The least fixed-point is already an infinite set! How are we going to
  construct a larger fixed-point?
\end{frame}
%%
\begin{frame}{Greatest Fixed Point}
  Let $\Semantic{\text{Any}}$ be the set of all values in our language.
  \medskip
  
  To crack a guess at what the greatest fixed point is let's iteratively apply
  $F$ to $\Semantic{\text{Any}}$ and see which values are eventually killed off.
  \medskip

  Observe that any value in the LFP will survive intact. We don't need to worry
  about those.
  \medskip

  I think it's fair to say that any value of the form
  ${\text{inj2}}^n\text{(x)}$, where $n \geq 0$, $n$ is as large as possible and
  $x$ is not $\text{inj1(null)}$, won't be in $F^{(n+1)}(\Semantic{\text{Any}})$
  \medskip

  Suppose, for a moment, that  we're allowed to have ``infinite'' values, like
  ${\text{inj2}}\ldots$. This value is special because
  $\text{inj2}({\text{inj2}}^\infty) = {\text{inj2}}^\infty$
  \medskip

  $\text{LFP} \cup \{{\text{inj2}}^\infty\}$ is the greatest fixed-point of $F$
\end{frame}
%%
\begin{frame}[fragile]{Lazy vs Strict -- 1}
  Infinite values aren't really a stretch when we're talking about programming
  languages. All we need is the ability to evaluate code lazily. Here's
  ${\text{inj2}}^\infty$ in Haskell, where's it's recognized as a Natural.
  (Given this, the type should more accurately be called ``Conatural'')
  %
\begin{verbatim}
module RecursiveTypes where

data Natural = Inj1 () | Inj2 Natural 

fix :: (t -> t) -> t
fix f = f (fix f)

inj2_infty :: Natural
inj2_infty = fix Inj2
\end{verbatim}
\end{frame}
%%
\begin{frame}[fragile]{Lazy vs Strict -- 2}
  Typed Racket uses strict semantics so we can't pull the same trick twice. Even
  though the recursive type Natural theoretically admits infinite values, we
  can't actually create one.
  \medskip

  We're aware that we can delay evaluation using thunks. Let's define a new type
  that supports delayed evaluation.
  %
\begin{verbatim}
(define-type Conatural (-> Null (Sum Null Conatural)))
\end{verbatim}
%
Note that the generating function for this recursive type is non-trivially
different from that of Natural, but I'm certain the information content is
the same as Haskell's Natural type. We can now represent infinity.
%
\begin{verbatim}
(define inj2_infty : Conatural 
  (lambda ([_ : Null]) (inj2 inj2_infty)))
\end{verbatim}
\end{frame}
%%
\begin{frame}{References}
  Chapters 20 and 21 of ``Types and Programming Languages'' by Benjamin C.
  Pierce
  \medskip

  \href{https://homepage.cs.uiowa.edu/~astump/notes/recursive-types-and-fixed-points.pdf}
       {A Note on Recursive Types and Fixed Points} by Aaron Stump
  \medskip

  \href{https://www.cs.cornell.edu/courses/cs4110/2018fa/lectures/lecture08.pdf}
       {Cornell CS 4110 -- Denotational Semantics Examples}
  \medskip

  Recommended during talk:
  \href{https://www.researchgate.net/publication/2615046_Calculating_Functional_Programs}
       {Calculating Functional Programs} by Jeremy Gibbons
\end{frame}
\end{document}
